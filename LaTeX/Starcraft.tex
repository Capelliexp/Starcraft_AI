\documentclass[a4paper,11pt]{article}

\usepackage[T1]{fontenc}	      %font - (base) HA ALLTID MED
\usepackage{lmodern}						%font - Standard
\usepackage[swedish]{babel}   %svenska
\usepackage[utf8]{inputenc}   %svenska åäö
\usepackage{lipsum}           %onödiga texten
\usepackage{booktabs}         %referat
\usepackage{amsmath, amssymb, upref} %matte
\usepackage{amsthm}           %omgivningar
%---
\usepackage{caption}
\usepackage{subcaption}	%använd antingen cap & subcap ELLER hyperref
%---
\usepackage{tocbibind}        %till referenser i innehållsförteckning
\usepackage{graphicx}         %till implementering av bilder
\usepackage{color}						%för text i färg
\usepackage[framemethod=tikz]{mdframed}	%highlighting hela stycken
\usepackage{listings}	%för kod
\usepackage{lr-cover}         %Roberts förstasida
\usepackage{labrapport}				%Roberts rapportmall

\usepackage{setspace}	%line space
	\singlespacing
	%\onehalfspacing
	%\doublespacing

\definecolor{dkgreen}{rgb}{0,0.6,0}		%för kod
\definecolor{gray}{rgb}{0.5,0.5,0.5}	%för kod
\definecolor{mauve}{rgb}{0.58,0,0.82}	%för kod

\lstset{	%för kod
	frame=tb,
  language=c++,
  aboveskip=3mm,
  belowskip=3mm,
  showstringspaces=false,
  columns=flexible,
  basicstyle={\small\ttfamily},
  numbers=none,
  numberstyle=\tiny\color{gray},
  keywordstyle=\color{blue},
  commentstyle=\color{dkgreen},
  stringstyle=\color{mauve},
  breaklines=true,
  breakatwhitespace=true,
  tabsize=3
}

\newcommand{\highlight}[1]{\colorbox{yellow}{#1}}	%highlighting små stycken

\long\def\*#1*/{}	%kommentarer - \* nu skriver jag en kommentar */

\begin{document}

\title{Game Artificial Intelligence \\ DV1569 \\ Project}
\author{Filip Pentikäinen}
\date{\today}
\maketitle

\section{Implementation}
Då min första prioritet var att se till att AI:n byggde upp de byggnader som var krav för att klara uppgiften, så var det också det som implementerades först. Varje steg och delsteg beskrevs som en \textit{task}, där varje task har ett ID, en enhet som ska konstrueras (både byggnader och gubbar), vilken bas som enheten ska skapas i, och räknare till utvecklingen. Dessa tasks blev en lång lista som AI:n kommer avklara successivt (se Figur 1).

När soldaterna senare skulle anfalla motståndaren var de tvungna att först hitta alla fiende baser. En lista skapas med alla potentiella baser och sedan söker soldaterna igenom varje bas tills fiender hittas. När en fiende byggnad väl hittas kommer alla enheters offensiva \textit{state} ta dom till den bas där fienden har hittats. När en sådan bas väl blir utplånad kommer sökandet fortsätta, tills spelet är vunnet.

\section{Svårigheter}
En av de svåraste stegen i det här projektet var att hantera \textit{BWTA} och dess struktur när man försöker extrahera data från regioner. Jag har inte tidigare hållit på mycket med \textit{set}, \textit{pair} eller \textit{map} så att lära sig hur dessa bör (och inte bör) används tog sin tid. 

Ett annat svårt steg var att skapa en algoritm som hittar en lämplig byggplats i en bas. Den slutgiltiga algoritmen söker i ett gridnät runt basens huvudbyggnad för att hitta en plats som passar, och som inte stänger in angränsande byggnader.

\section{Efterkloka Råd}
När bot:en skapades var jag medveten om mycket ''ful'' programmering som jag gjorde, för jag hade inte kunskap nog för att komma på den optimala lösningen. Om jag i framtiden skapar en till RTS-bot kommer jag börja med att konstant hålla flera listor uppdaterade. Dessa listor kommer innehålla basinformation, synliga fiende enheter, smarta armeplaceringar m.m.. Det tillvägagångssättet hade gjort det lättare att ta beslut angående truppers rörelse och prioriteringsförmåga. Just nu har jag ingen sådan omfattande lista vilket gör att i flera funktioner så måste jag skapa och söka igenom listor som tidigare skapats, men som gått ur scope.

\section{Avslut}
Jag tycker dokumentationen till BWAPI och BWTA har varit lite snål med information, men annars varit till stor hjälp. Detta har varit ett riktigt roligt projekt, och det har gett mig välbehövd insikt i vilka problem som kan uppstå om man inte skapar denna typen av program rätt från början. 

\begin{figure}[b]
\begin{lstlisting}
	task[0] =  { 1, 1, 0, ...::Terran_Supply_Depot, 2, 0, 0 };
	task[1] =  { 1, 2, 0, ...::Terran_Barracks, 1, 0, 0 };
	task[2] =  { 1, 3, 0, ...::Terran_Marine, 10, 0, 0 };
	...
\end{lstlisting}
\caption{Varje delstegs \textit{task}.}
\end{figure}

\end{document} %----------------------------------------------------------------

\*

HUR MAN SKAPAR EN TABELL:
\begin{tabular}{|l|c|c|c|c|}
\label{table_1}
\hline
& { Union: 100} & { Union: 1000 } & { Union: 5000 } & { Union: 9000 } \\
\hline
find och unionSets 							& 1.66 & 3.67 & 33.77 & 1356.62 \\
find och unionSetsRanks 				& 1.01 & 2.00 & 5.45 	& 6.6 		\\
findCompress och unionSets 			& 1.66 & 3.29 & 9.33 	& 8.5 		\\
findCompress och unionSetsRanks & 1.01 & 2.00 & 4.75 	& 4.21 		\\
\hline
\end{tabular}

HUR MAN INFOGAR EN BILD (som ligger i samma mapp som .tex):
\begin{figure}[t]
\begin{center}
\label{bild_1}
\includegraphics[width=7cm]{bild.jpg}
\caption{stuff under bilden...}
\end{center}
\end{figure}

HUR MAN MANUELLT FLYTTAR PÅ SAKER (kommer troligen ge varning):
\vskip+0.1cm
\hskip+0.1cm

HUR MAN REFERERAR (funkar till bilder, tabeller, böcker, kapitel m.m.):
\ref(label_x)

BOLD OCH ITALIC:
textbf{bold text}
textit{italic text}

HUR MAN SKRIVER SIFFROR/FORMLER:
 $x+5 = 10$
 se https://en.wikibooks.org/wiki/LaTeX/Mathematics#Operators

HUR MAN SKAPAR LISTOR(punkt, nummer eller desc):
https://en.wikibooks.org/wiki/LaTeX/List_Structures#List_structures

*/